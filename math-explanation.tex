\documentclass[12pt]{article}

% Packages
\usepackage{amsmath} % For mathematical symbols and environments
\usepackage{amsfonts} % For math fonts
\usepackage{amssymb} % For additional symbols
\usepackage{graphicx} % For including images
\usepackage{hyperref} % For hyperlinks
\usepackage{geometry} % For customizing page margins
\usepackage{fancyhdr} % For custom headers and footers

% Page setup
\geometry{margin=1in} % Customize margins if needed

% Header and Footer
\pagestyle{fancy}
\fancyhf{}
\lhead{Patrick Su}
\rhead{Math Blog Post}
\cfoot{\thepage}

% Title
\title{x-power-x}
\author{Patrick Su}
\date{\today}

\begin{document}

\maketitle

\begin{abstract}

\end{abstract}

\section{Introduction}
% Provide an introduction to your topic and what readers can expect from the blog post.
Welcome to my math blog! Today, we'll dive into some captivating mathematical ideas and explore their foundations with rigorous mathematics.

\section{Concept 1: [Concept Name]}
% Explain the first concept with rigorous mathematical explanations.
In this section, we will delve into the fascinating world of [Concept Name] and rigorously prove its fundamental properties.

\section{Concept 2: [Concept Name]}
% Explain the second concept with rigorous mathematical explanations.
Next, we'll explore [Concept Name] and unveil its underlying mathematical principles through clear proofs.

\section{Applications and Examples}
% Showcase some real-world applications or interesting examples related to the concepts.
To better grasp these concepts, let's look at some real-world applications and intriguing examples.

\section{Conclusion}
% Summarize the key points and highlight the significance of the concepts discussed.
In conclusion, we have explored two captivating mathematical concepts, [Concept 1] and [Concept 2], and demonstrated their rigorous mathematical foundations. Mathematics continues to be a boundless realm of curiosity and discovery.

% You can add more sections if you wish, or include figures, references, etc.

\end{document}
